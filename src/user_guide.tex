\documentclass[11pt,a4paper]{article}

\usepackage{pslatex}
\usepackage{inconsolata}
\usepackage[square,numbers]{natbib}
\usepackage[includehead,top=2.5cm,bottom=2.5cm,
            left=3cm,right=3cm]{geometry}
\usepackage{parskip}
\usepackage{amsmath}
\usepackage{amssymb}

\usepackage{color}
\definecolor{bluish}{rgb}{0.20,0.29,0.46}
% *always \use this last*
\usepackage[colorlinks,breaklinks,pdftex,bookmarks=true,
            linkcolor=bluish,citecolor=bluish,urlcolor=bluish]{hyperref}

\usepackage{enumitem}
\setlist{topsep=0ex,itemsep=2pt,partopsep=0pt,parsep=0pt,leftmargin=7.5mm}

% Bold face for vectors
\renewcommand{\vec}[1]{\mathbf{#1}}

% Sublist style for drafting
\renewcommand\labelenumii{\theenumii.}
\renewcommand\theenumii{\arabic{enumii}}
\renewcommand\theenumiii{\arabic{enumiii}}

\begin{document}

{\huge User Guide for Spowtd v0.0.0}\\[1ex]

\renewcommand{\baselinestretch}{1.18}\normalsize

This is the user guide for Spowtd, which implements the scalar
parameterization of water table dynamics described in
\citet{Cobb_et_al_2017}.

\section{The steps of scalar parameterization}
Scalar parameterization involves these essential steps:
\begin{enumerate}
\item Load water level, precipitation and evapotranspiration data;
\item Identify dry intervals and storm intervals;
\item Match intervals of rising water levels to rainstorms;
\item Construct a master rising curve;
\item Construct a master recession curve;
\item Fit a preliminary specific yield function to the master rising
  curve;
\item Jointly fit a specific yield and a conductivity (equivalently,
  transmissivity) function to the master rising and recession curves.
\end{enumerate}

\section{The spowtd script}
The \texttt{spowtd} script provides a command-line interface to
perform calculations with Spowtd.  It has these subcommands:
\begin{itemize}
\item \texttt{spowtd load}\ldots
\item \texttt{spowtd classify}\ldots
\item \texttt{spowtd recession}\ldots
\item \texttt{spowtd rise}\ldots
\item \texttt{spowtd simulate}\ldots
\end{itemize}

\bibliographystyle{unsrtnat}
\bibliography{user_guide.bib}

\end{document}
