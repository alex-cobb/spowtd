\documentclass[11pt,a4paper]{article}

\usepackage{pslatex}
\usepackage{inconsolata}
\usepackage[square,numbers]{natbib}
\usepackage[includehead,top=2.5cm,bottom=2.5cm,
            left=3cm,right=3cm]{geometry}
\usepackage{listings}
\usepackage{parskip}
\usepackage{amsmath}
\usepackage{amssymb}

\usepackage{color}
\definecolor{bluish}{rgb}{0.20,0.29,0.46}
% *always \use this last*
\usepackage[colorlinks,breaklinks,pdftex,bookmarks=true,
            linkcolor=bluish,citecolor=bluish,urlcolor=bluish]{hyperref}

\usepackage{xcolor}

\definecolor{codegreen}{rgb}{0,0.6,0}
\definecolor{codegray}{rgb}{0.5,0.5,0.5}
\definecolor{codepurple}{rgb}{0.58,0,0.82}
\definecolor{backcolour}{rgb}{0.95,0.95,0.92}

\lstdefinestyle{mystyle}{
    backgroundcolor=\color{backcolour},
    commentstyle=\color{codegreen},
    numberstyle=\tiny\color{codegray},
    stringstyle=\color{codepurple},
    basicstyle=\ttfamily\footnotesize,
    breakatwhitespace=false,
    breaklines=true,
    captionpos=b,
    keepspaces=true,
    numbers=left,
    numbersep=5pt,
    showspaces=false,
    showstringspaces=false,
    showtabs=false,
    tabsize=2
}

\lstset{style=mystyle}

\usepackage{enumitem}
\setlist{topsep=0ex,itemsep=2pt,partopsep=0pt,parsep=0pt,leftmargin=7.5mm}

% Bold face for vectors
\renewcommand{\vec}[1]{\mathbf{#1}}

% Sublist style for drafting
\renewcommand\labelenumii{\theenumii.}
\renewcommand\theenumii{\arabic{enumii}}
\renewcommand\theenumiii{\arabic{enumiii}}

\begin{document}
\lstset{language=bash}

{\huge User Guide for Spowtd v0.2.0}\\[2ex]
{\large Alex Cobb}\\[0ex]

\renewcommand{\baselinestretch}{1.18}\normalsize

This is the user guide for Spowtd, which implements the scalar
parameterization of water table dynamics described in
\citet{Cobb_et_al_2017} and \citet{Cobb_and_Harvey_2019}.

\section{The steps of scalar parameterization}
Scalar parameterization involves these essential steps:
\begin{enumerate}
\item Load water level, precipitation and evapotranspiration data;
\item Identify dry intervals and storm intervals;
\item Match intervals of rising water levels to rainstorms;
\item Construct a master rising curve;
\item Construct a master recession curve;
\item Fit a preliminary specific yield function to the master rising
  curve;
\item Jointly fit a specific yield and a conductivity (equivalently,
  transmissivity) function to the master rising and recession curves.
\end{enumerate}

Steps 6 and 7 are not implemented yet.

\section{The spowtd script}
The \texttt{spowtd} script provides a command-line interface to
perform calculations with Spowtd.

\subsection{Dependencies}
Running the script requires Python 3 and the Python packages
\href{https://matplotlib.org/}{Matplotlib},
\href{https://numpy.org/}{Numpy}, and
\href{https://pypi.org/project/pytz/}{Pytz}.

\subsection{Using the script}
The \texttt{spowtd} script has these subcommands (typically run in
this order):
\begin{itemize}
\item \texttt{spowtd load}: Load water level, precipitation and
  evapotranspiration data
\item \texttt{spowtd classify}: Classify data into storm and
  interstorm intervals
\item \texttt{spowtd set-zeta-grid}: Set up water level grid for
  master curves
\item \texttt{spowtd recession}: Assemble recession curve
\item \texttt{spowtd rise}: Assemble rise curve
\item \texttt{spowtd plot}: Plot data
\item \texttt{spowtd simulate}: Simulate data rise curve, recession
  curve, or rising and receding intervals.
\end{itemize}

The first step is to load the precipitation, evapotranspiration and
water level data.  The input text files must be in an UTF-8-compatible
encoding (ASCII is fine).  The time zone is stored with the dataset
and will be used in plots (all times are stored internally as UNIX
timestamps).  For example, to load data into a new dataset file called
\texttt{ekolongouma.sqlite3}:
\begin{lstlisting}[frame=single]
spowtd load ekolongouma.sqlite3 \
  -vvv \
  --precipitation src/precipitation_Ekolongouma.txt \
  --evapotranspiration src/evapotranspiration_Ekolongouma.txt \
  --water-level src/waterlevel_Ekolongouma.txt \
  --timezone Africa/Lagos
\end{lstlisting}
The verbosity flags (\texttt{-vvv}) are not required; they cause the
script to report more on what is being done.

Next, classify the water level and precipitation time series into
storm and interstorm intervals based on thresholds for rainfall
intensity and rates of increase in water level.  For example, this
command classifies intervals with precipitation of at least 4~mm / h
as storms, and intervals in which the water level is increasing at a
rate of least 8~mm / h as storm response.
\begin{lstlisting}[frame=single]
spowtd classify ekolongouma.sqlite3 \
  -vvv \
  --storm-rain-threshold-mm-h 4.0 \
  --rising-jump-threshold-mm-h 8.0
\end{lstlisting}

At this stage the classification can be plotted.  A basic interactive
plot showing the classified water level and precipitation time series
can be produced with:
\begin{lstlisting}[frame=single]
spowtd plot time-series ekolongouma.sqlite3
\end{lstlisting}
The parts of the water level time series marked as interstorms are on
a light red background, and the parts of the water level time series
marked as storm response are on a light green background.  The parts
of the precipitation time series marked as storms are on a light blue
background.  You can pan in the plot with the right mouse button and
zoom with a left mouse button, or use the magnifying glass to zoom in.
You can revert to earlier zoom and pan values with the arrow buttons.

Adding \texttt{-f} or \texttt{--flags} highlights the parts of the
water level time series that have been classified as storm response
and interstorms, and the parts of the precipitation time series
\begin{lstlisting}[frame=single]
spowtd plot time-series ekolongouma.sqlite3  -f
\end{lstlisting}
The rising intervals are highlighted in blue, intervals with rising
intervals that could not be matched to rain storms are highlighted in
magenta, and rain storms are highlighted in red.

The next step is to establish a uniform grid for water levels.  This
grid is used when storm and interstorm intervals are assembled into
rising and recession curves.
\begin{lstlisting}[frame=single]
spowtd set-zeta-grid -vvv ekolongouma.sqlite3
\end{lstlisting}

The next two steps assemble the recession and rise curves:
\begin{lstlisting}[frame=single]
spowtd recession -vvv ekolongouma.sqlite3
\end{lstlisting}

\begin{lstlisting}[frame=single]
spowtd rise -vvv ekolongouma.sqlite3
\end{lstlisting}

The recession and rise curves are now assembled, and can be plotted.
\begin{lstlisting}[frame=single]
spowtd plot recession ekolongouma.sqlite3
\end{lstlisting}

\begin{lstlisting}[frame=single]
spowtd plot rise ekolongouma.sqlite3
\end{lstlisting}
These plots can be interacted with in the same way: left mouse button
to pan, right mouse button to zoom, disk icon to save.

\section{Parameterization}
Parameters are provided to \texttt{spowtd} in
\href{https://yaml.org/}{YAML} format.

Currently two types of parameter sets are supported: (1) Cubic spline
for specific yield, piecewise linear for the logarithm of
conductivity; and (2) The PEATCLSM parameterization.

The spline parameterizations look like this:
\begin{lstlisting}[frame=single]
specific_yield:
  type: spline
  zeta_knots_mm:
    - -291.7
    - -183.1
    - -15.74
    - 10.65
    - 38.78
    - 168.3
  sy_knots:  # Specific yield, dimensionless
    - 0.1358
    - 0.1671
    - 0.2541
    - 0.2907
    - 0.2892
    - 0.6857
transmissivity:
  type: spline
  zeta_knots_mm:
    - -291.7
    - -5.167
    - 168.3
    - 1000
  K_knots_km_d:  # Conductivity, km/d
    - 5.356e-3
    - 1.002
    - 6577.0
    - 8.430+03
  minimum_transmissivity_m2_d: 7.442  # Minimum transmissivity, m2/d
\end{lstlisting}
and the PEATCLSM parameterizations look like this:
\begin{lstlisting}[frame=single]
specific_yield:
  type: peatclsm
  sd: 0.162  # standard deviation of microtopographic distribution, m
  theta_s: 0.88  # saturated moisture content, m^3/m^3
  b: 7.4  # shape parameter, dimensionless
  psi_s: -0.024  # air entry pressure, m
transmissivity:
  type: peatclsm
  Ksmacz0: 7.3  # m/s
  alpha: 3  # dimensionless
  c: 1.5e-5  # m^-1
\end{lstlisting}
(the text following each parameter, after the \verb|#|, is a comment
and invisible to spowtd).

The specific yield and transmissivity curves can be plotted with
\begin{lstlisting}[frame=single]
  spowtd plot WHAT parameters.yml
\end{lstlisting}
where \texttt{WHAT} is one of \texttt{specific-yield},
\texttt{conductivity} or \texttt{transmissivity} and
\texttt{parameters.yml} is a YAML file containing hydraulic
parameters.

In addition, the plotting commands \texttt{plot rise}, \texttt{plot
  recession} and \texttt{plot time-series} support a parameter
\texttt{-p}, \texttt{--parameters}; if a YAML file containing
hydraulic parameters is passed to one of these commands, the
corresponding plot (rising curve, recession curve, rising and receding
intervals) is simulated using those parameters.

The simulated curves and corresponding data can be obtained as text
using \texttt{spowtd simulate WHAT data.sqlite3 parameters.yml} where
\texttt{WHAT} is \texttt{rise}, \texttt{recession}, or
\texttt{intervals}.  These commands write simulated data, water level
data, and / or residuals to an output file (standard output by
default) as delimited text.  For example,
\begin{lstlisting}[frame=single]
spowtd simulate rise ekolongouma.sqlite3 parameters.yml
\end{lstlisting}
reads data from \texttt{ekolongouma.sqlite3} and parameters from the
file \texttt{parameters.yml} and writes the assembled and simulated
rise curves to standard output.

\bibliographystyle{unsrtnat}
\bibliography{user_guide.bib}

\end{document}
